\documentclass[twocolumn, a4paper]{jsarticle}
\usepackage[top=20truemm,bottom=20truemm,left=20truemm,right=20truemm]{geometry}
\usepackage[dvipdfmx]{graphicx,color}
\usepackage{cite}
\usepackage{url}

\setlength{\columnsep}{5mm}

\begin{document}
\title{情報システム工学演習II 画像処理 レポートテンプレート}
\author{08D12345 大倉 史生}
\date{20xx年x月x日} 
\twocolumn[
\maketitle
]

これは、情報システム工学演習II 画像処理演習のレポートテンプレートである。
\begin{itemize}
\item 本レポートの作成には\TeX を使う。
\item ページ数は、参考文献リストを除いて1~2ページ程度を目安とする(が、それより長くても良い)。日本語か英語で記述すること。
\item 実装したアプリの内容をうまくアピールするように、本レポートのタイトルを適切に変更すること。
\item 書き上げたレポートをコンパイルし、\{学籍番号\}.pdfのファイル名で提出すること。その際、この辺のインストラクション用の文章は削除すること。
\end{itemize}
本演習は、アプリの「独創性」および「完成度」、レポート記述の「充実度」で評価する。
下記に章立ての一例を示す。この順番にこだわる必要はないが、概ね以下に示すような内容を記載するのが望ましい。

\section{背景}
この世界、あるいは皆さんの生活にはどのような(潜在的あるいは顕在的な)問題があり、この演習で作ったシステム、ツール、アプリがどのように役に立つか説明すること。
ここでは、あなた自身、あるいはターゲットとなるユーザにとっての有用性がきちんと説明されていれば十分である。
一般的な観点では有用でなくとも、あなた自身(あるいは特定のユーザ)が有用であると思い得ることがレポートの記述からわかれば、それで良い。
例えば、「暇つぶし」や「楽しい」なども立派な用途である。これらが有用となり得るシナリオをレポート中にしっかり記載すること。

\section{実装したツール}
本演習で実装したツールの概要を記述する。

\subsection{外部仕様}
ユーザはどのようにそのツールを使うのか記述する。どんな画像を用意して、何を起動する、どのボタンを押す、そうすると何が得られる、など。

\subsection{内部仕様}
そのツールは、どのような仕様で実装されたか説明する。処理のフローなどを示すのも有用。参考にした資料がある場合は引用すること(論文の引用例\cite{canny}、URLでの引用例\cite{material})。

\subsection{使用した関数やライブラリ・実行環境}
デフォルト以外のライブラリを使った場合、また、特殊な環境で実行することを前提とする場合は記載すること。

\begin{figure}[t]
\centering
\includegraphics[width=\linewidth]{ou.png}
\caption{図の例}
\label{fig:ou}
\end{figure}

\section{実行結果}
実行例を記載する。実験用素材(入力画像など)は、構築したアプリの機能をうまくアピールできるよう、自分で撮影(あるいは収集)することが望ましい。
実際の入出力画像を図として貼り付けるのが有用であるが、どのような結果が得られるのかを文章でも説明することが重要である。
図\ref{fig:ou}に図の貼り付け例を示す。
うまくいく例・うまくいかない例を示し、どのようにすれば改善できそうかを次節で考察できると、さらに良い。


\section{考察・感想}
考察や感想を記載する。

\begin{thebibliography}{99}
\bibitem{canny} John Canny. A computational approach to edge detection. {\em IEEE Transactions on Pattern Analysis and Machine Intelligence},  No.~6, pp. 679--698, 1986.
\bibitem{material} 情報システム工学演習II画像処理 演習資料, \url{https://github.com/fumio125/enshu_ip}.
\end{thebibliography}


\begin{figure}[t]
\centering
\includegraphics[width=\linewidth]{overleaf.png}
\caption{Overleafの設定}
\label{fig:overleaf}
\end{figure}

\section*{Appendix: \LaTeX 環境構築}
\subsection*{クラウドツール}
Overleaf\footnote{Overleaf, \url{https://ja.overleaf.com/}}などのクラウドツール使うと、手元の環境構築が必要ないため便利である。\texttt{report}フォルダをzip圧縮して、「プロジェクトのアップロード」をすると編集できる。
日本語の文書のコンパイルには、図\ref{fig:overleaf}に示すような設定変更が必要であり、メニューから、コンパイラをLaTeXに変更してコンパイルすると良い。
なお、必要な項目を記載した\texttt{latexmkrc}も用意する必要があるが、すでに\texttt{report}フォルダ内に含まれているので改めて追加する必要はない。

\subsection*{自前環境}
手元に環境構築する場合はTeX Live\footnote{Tex Live, \url{https://www.tug.org/texlive/}}を使うと良い。インストール方法は\TeX WikiのTeX Liveのページ\footnote{\TeX Wiki, \url{https://texwiki.texjp.org/?TeX%20Live}}が詳しい。 
日本語の文章なので、pLaTeX(など)でコンパイルすること。

TeX Liveと一緒にインストールされる編集ソフト(TeXworks)を使う場合、図\ref{fig:texworks}のようにコンパイルツールをpLaTeXにすると良い(「再生ボタン」あるいはWindowsの場合はCtrl+Tでコンパイルできる)。
参考文献と本文中の文献番号、図表の番号などを対応付けるためには、何回か走らせる(本文中の?が消えるまで;普通は2~3回で良い)必要がある。

\begin{figure}[t]
\centering
\includegraphics[width=\linewidth]{texworks.png}
\caption{TeXworksの設定}
\label{fig:texworks}
\end{figure}


\end{document}